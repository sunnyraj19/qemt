\
\documentclass[12pt]{article}
\usepackage{amsmath, amssymb, graphicx, hyperref}
\title{Quantum Error Mitigation Toolkit: Methods and Benchmarks}
\author{Your Name}
\date{\today}

\begin{document}
\maketitle

\begin{abstract}
We present QEMT, an open-source Python toolkit implementing and benchmarking error-mitigation techniques for noisy quantum devices, focusing on Zero-Noise Extrapolation (ZNE), Dynamical Decoupling (DD), Measurement Error Mitigation (EMM), and Clifford Data Regression (CDR).
\end{abstract}

\section{Introduction}
Brief overview of NISQ era, motivation for error mitigation, and contributions.

\section{Methods}
\subsection{Zero-Noise Extrapolation}
Explain circuit folding, scaling factors, and fitting.

\subsection{Dynamical Decoupling}
Describe XY4, CPMG sequences and insertion via transpiler passes.

\subsection{Measurement Error Mitigation}
Outline calibration circuits and correction matrix.

\subsection{Clifford Data Regression}
Summarize regression approach.

\section{Benchmarks}
\subsection{Grover Search}
Present results with/without ZNE.

\subsection{VQE(H$_2$)}
Energy error reduction with ZNE.

\subsection{QAOA(MaxCut)}
Improvement with DD.

\section{Conclusion}
Summarize findings and future work.

\bibliographystyle{plain}
\bibliography{refs}

\end{document}
