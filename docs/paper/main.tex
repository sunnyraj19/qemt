\
\documentclass[11pt]{article}
\usepackage[margin=1in]{geometry}
\usepackage{amsmath,amssymb,graphicx,hyperref}
\title{Error Mitigation on NISQ Devices: A Practical Toolkit and Benchmarks}
\author{Your Name}
\date{\today}
\begin{document}
\maketitle

\begin{abstract}
We implement and benchmark a compact toolkit of error-mitigation methods---Zero-Noise Extrapolation (ZNE), measurement mitigation, and dynamical decoupling---on Grover, VQE(H$_2$), and QAOA(MaxCut). We show consistent reductions in observable error on simulated noisy backends and provide optional hooks for IBM devices.
\end{abstract}

\section{Introduction}
NISQ devices are limited by noise; error mitigation offers a near-term path to improved expectation values without full fault tolerance.

\section{Methods}
\subsection{Zero-Noise Extrapolation}
Circuit folding with odd scale factors and Richardson extrapolation to estimate the zero-noise observable.

\subsection{Measurement Mitigation}
Calibration-based readout error mitigation.

\subsection{Dynamical Decoupling}
Padding idle periods with XY4/CPMG sequences to reduce decoherence.

\section{Experiments}
\paragraph{Grover + ZNE} Report baseline vs. ZNE-extrapolated observables.
\paragraph{VQE(H$_2$)} Compare noisy energy, ideal energy, and ZNE-mitigated energy.
\paragraph{QAOA(MaxCut)} Show best cut value before/after DD.

\section{Results}
Insert figures exported from \texttt{experiments/results}.

\section{Discussion and Future Work}
Combine ZNE with DD; extend to larger qubit counts and additional ans\"atze.

\bibliographystyle{plain}
\bibliography{refs}
\end{document}
